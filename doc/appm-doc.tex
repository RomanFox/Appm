\documentclass{article}

\usepackage[utf8]{inputenc}
\usepackage[T1]{fontenc}
\usepackage{amsmath,amssymb}
\usepackage{url}


\usepackage{lmodern}




\title{Source code documentation of APPM}
\author{Roman Fuchs}

\begin{document}

\maketitle

\tableofcontents

\vspace{2em}

APPM: asymptotic preserving plasma model.

\section{Introduction}

Aim of the code: show the feasibility of a plasma model that is based 
on the Maxwell Grid Equations (see Finite Integration Technique) for 
electromagnetism and the Navier-Stokes equations for the fluid. 

Equations:



\section{Mesh construction}

Why a primal and dual mesh?


\subsection{Primal mesh}

How it is defined.



\subsection{Dual mesh}

How it is defined.



\section{Data output}

The data is visualized in ParaView\footnote{version 5.6.0, 64-bit} 
using XDMF\footnote{\url{xdmf.org/index.php/XDMF_Model_and_Format}, version 3.} 
for data description and HDF5\footnote{version 1.10, 64-bit} 
for the heavy data.

Remark: instead of ParaView, one could also use VisIT for visualization. However, it does not support polygonal cells. 






\subsection{Mesh}

Definition of cells and faces as given in the XDMF format. 

For each face: facetype + list of vertex indices. Except for a polygon: facetype + number of vertices + list of vertex indices.

For each cell: celltype + list of vertex indices. Except for a polyhedral: celltype + number of faces + description of each face.




\subsection{Data}







\end{document}
