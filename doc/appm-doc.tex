\documentclass{article}

\usepackage[utf8]{inputenc}
\usepackage[T1]{fontenc}
\usepackage{amsmath,amssymb}
\usepackage{url}
\usepackage{todonotes}


\usepackage{lmodern}

\let\oldeps=\epsilon
\renewcommand{\epsilon}{\varepsilon}

% Vector letters
\newcommand{\vecLetter}[1]{\vec{#1}}
\newcommand{\Bvec}{\vecLetter{B}}
\newcommand{\Dvec}{\vecLetter{D}}
\newcommand{\Evec}{\vecLetter{E}}
\newcommand{\Hvec}{\vecLetter{H}}
\newcommand{\Jvec}{\vecLetter{J}}




\title{Source code documentation of APPM}
\author{Roman Fuchs}

\begin{document}

\maketitle

\tableofcontents

\listoftodos

\vspace{2em}

APPM: asymptotic preserving plasma model.

\section{Introduction}

Aim of the code: show the feasibility of a plasma model that is based 
on the Maxwell Grid Equations (see Finite Integration Technique) for 
electromagnetism and the Navier-Stokes equations for the fluid. 

Maxwell equations:

\begin{subequations}
\begin{equation}
\partial_t \Bvec + \nabla \times \Evec = 0
\end{equation}
\begin{equation}
\partial_t \Dvec - \nabla \times \Hvec = -\Jvec
\end{equation}
\begin{equation}
\nabla \cdot \Bvec = 0
\end{equation}
\begin{equation}
\nabla \cdot \Dvec = \rho
\end{equation}
($\rho$: space charge density)
\begin{equation}
\Dvec = \epsilon \Evec
\end{equation}
\begin{equation}
\Bvec = \frac{1}{\mu} \Hvec
\end{equation}
\end{subequations}





\section{Mesh construction}

Why a primal and dual mesh?


\subsection{Primal mesh}

How it is defined.

Circumcenter of a triangle face: \\




\subsection{Dual mesh}

How it is defined.



\section{Data output}

The data is visualized in ParaView\footnote{version 5.6.0, 64-bit} 
or VisIT, 
using XDMF\footnote{\url{xdmf.org/index.php/XDMF_Model_and_Format}, version 3.} 
for data description and HDF5\footnote{version 1.10, 64-bit} 
for the heavy data.

Remarks:
\begin{itemize}
\item Use ParaView, because VisIT does not support polyhedral cells.

\item ParaView has three readers for *.xdmf-files: XMDF Reader, Xdmf3ReaderS, and Xdmf3ReaderT. 

\item XDMF Reader supports the \texttt{GridType=''Tree''} tag (i.e., grid of grids), but does not support polyhedrals.

\item Xdmf3ReaderS supports polyhedrals, but does not support \texttt{GridType=''Tree''}.

\item Therefore, the mesh data is split into two xdmf files: one that has vertex, edge, and surface data (that can be read with XDMF Reader); and a separate one for volume data (that can be read with Xdmf3ReaderS).

\end{itemize}







\subsection{Mesh}

Definition of cells and faces as given in the XDMF format. 

For each face: facetype + list of vertex indices. Except for a polygon: facetype + number of vertices + list of vertex indices.

For each cell: celltype + list of vertex indices. Except for a polyhedral: celltype + number of faces + description of each face.




\subsection{Data}


\section{Visualization}

\begin{itemize}
\item Use ParaView
\item Open file \texttt{appm.xdmf}
\item Select Reader \texttt{XDMF Reader} (other options: XMDF Reader S, XDMF Reader T; they do not work)

\item To visualize data on a given mesh: add Filter, extract Block.
\end{itemize}


Vectors:

\begin{itemize}
\item Add Filter -> Glyphs
\item Orientation array = \ldots
\item Scale array = \ldots
\item Masking / Glyph mode = All points
\end{itemize}

\begin{description}
\item[Animation View, Time Inspector]
From the menu bar: View. 
\end{description}



\section{Testcases}


\subsection{Uniform current, determine magnetic fields}

Define current density in z-direction, at radius $r < r_0$. 





\section{TODO}


\todo[inline]{Raviart-Thomas interpolation of magnetic flux $B$}






\end{document}
